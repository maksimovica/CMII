\input{header.tex}
\usepackage[utf8]{inputenc}
\usepackage{amsmath}
\usepackage{amssymb}
\usepackage{amsfonts}
\usepackage{amssymb}
\usepackage{float}
\usepackage{indentfirst}
\usepackage{vmargin}
\usepackage{indentfirst}
\usepackage{titling}
\usepackage{color} 
\usepackage{siunitx}
\usepackage{xspace}
\usepackage{graphicx}
\usepackage{enumitem}
\usepackage[backend=biber,backref=true,style=unsrt,
style=numeric-comp,block=ragged,firstinits=true]{biblatex}
\addbibresource{ref-notes.bib}
\bibliography{ref-notes}
\graphicspath{{plot_synthesis/} {Feynman/}}

\newcommand{\mastersig}{\ensuremath{\Im{\widehat{\Sigma}^{A,B}(k,E)}}\xspace}
\newcommand{\chiqw}{\ensuremath{\Im{\chi}(q,\omega)}\xspace}

\providecommand{\norm}[1]{\lVert#1\rVert}

\newcommand{\subtitle}[1]{%
  \posttitle{%
    \par\end{center}
    \begin{center}\large#1\end{center}
    \vskip0.5em}%
}


\title{Condensed Matter II}
\subtitle{Problem set \#6}
%\author{}
\date{Spring 2014}

\begin{document}

\maketitle

\setlength{\unitlength}{1cm}
%\advance\textwidth by 3cm
%\advance\hoffset by -1.5cm 
\advance\textheight by 1cm
\advance\voffset by -1.5cm
\setmarginsrb{3cm}{0.5cm}{1.5cm}{1cm}{1cm}{1cm}{1cm}{1cm}
%\setlength{\parindent}{0cm}%

\pagestyle{plain}

\section{Cyclotron resonance of electrons and holes}

Extend the description of cyclotron resonance to anisotropic band
structure with the following dispersion relation

\[ E(k) = \dfrac{\hbar}{2}\biggl( \dfrac{k_x^2+k_y^2}{m_t} + \dfrac{k_z^2}{m_l}\biggr) \]

where $m)l$ and $m_t$ are the longitudinal and transversal effective
masses.

Hint: work in the semiclassical framework, and look for solutions of
the equation of motion 

\[ \dfrac{dP}{dt} = e \biggl( E_0e^{i\omega t}+\dfrac{v x
  H}{c}\biggr)\]
\[ P = p - e \dfrac{A}{c}\]

where $A$ is the potential vector. The effective Hamiltonian is given
by 

\begin{enumerate}[label=(\roman*)]
\item How many atoms per unit cell does the structure count?
\item Which are the Re energy bands, which are the O energy bands?
\item Is ReO$_3$ a metal, or a semiconductor?
\item Identify the location of the carrier pockets, and qualitatively assess the
  effective masses of the carriers.
\item Assess the number of electrons per carrier pocket.
\item Where does the lowest energy optical transition occur? Comment
  on its expected strength.
\end{enumerate}

%\section

\end{document}