\input{header.tex}
\usepackage[utf8]{inputenc}
\usepackage{amsmath}
\usepackage{amssymb}
\usepackage{amsfonts}
\usepackage{amssymb}
\usepackage{float}
\usepackage{indentfirst}
\usepackage{vmargin}
\usepackage{indentfirst}
\usepackage{titling}
\usepackage{color} 
\usepackage{siunitx}
\usepackage{xspace}
\usepackage{graphicx}
\usepackage{enumitem}
\usepackage[backend=biber,backref=true,style=unsrt,
style=numeric-comp,block=ragged,firstinits=true]{biblatex}
\addbibresource{ref-notes.bib}
\bibliography{ref-notes}
\graphicspath{{plot_synthesis/} {Feynman/}}

\newcommand{\mastersig}{\ensuremath{\Im{\widehat{\Sigma}^{A,B}(k,E)}}\xspace}
\newcommand{\chiqw}{\ensuremath{\Im{\chi}(q,\omega)}\xspace}

\providecommand{\norm}[1]{\lVert#1\rVert}

\newcommand{\subtitle}[1]{%
  \posttitle{%
    \par\end{center}
    \begin{center}\large#1\end{center}
    \vskip0.5em}%
}

\title{Condensed Matter II}
\subtitle{Problem set \#6}
%\author{}
\date{Spring 2014}

\begin{document}

\maketitle

\setlength{\unitlength}{1cm}
%\advance\textwidth by 3cm
%\advance\hoffset by -1.5cm 
\advance\textheight by 1cm
\advance\voffset by -1.5cm
\setmarginsrb{3cm}{0.5cm}{1.5cm}{1cm}{1cm}{1cm}{1cm}{1cm}
%\setlength{\parindent}{0cm}%

\pagestyle{plain}

\section{Cyclotron resonance of electrons and holes}

In this section, we extend the description of cyclotron resonance to an anisotropic band
structure with the following dispersion relation, with $m_l$ and $m_t$
the longitudinal and transversal effective masses. 

\[ E(\vec{k}) = \dfrac{\hbar^2}{2}\biggl( \dfrac{k_x^2+k_y^2}{m_t} + \dfrac{k_z^2}{m_l}\biggr) \]

The magnetic field
is assumed to lie in the $(x,z)$ plane, making an angle $\theta$ with the longitudinal axis:
\[ \vec{H} = H_0 (\sin \theta, 0, \cos \theta)\]

\begin{enumerate}[label=(\roman*)]
\item working in the semiclassical framework, we look for solutions of
the equation of motion 

\[ \dfrac{d\vec{P}}{dt} = e \biggl( \vec{E_0}e^{i\omega t}+\dfrac{\vec{v} \times
  \vec{H}}{c}\biggr)\]
\[ \vec{P} = \vec{p} - e \dfrac{\vec{A}}{c}\] where $\vec{A}$ is the potential vector.
\item The effective Hamiltonian is given by 
\[ H(\vec{P}) = \dfrac{1}{2} \biggl( \dfrac{P_x^2+P_y^2}{m_t} +
\dfrac{P_z^2}{m_l} \biggr)\]
\item The velocity vector is given by
\[ \vec{v} = \nabla_{\vec{P}} H(\vec{P}) = \biggl(\dfrac{P_x}{m_t}, \dfrac{P_y}{m_t}, \dfrac{P_z}{m_l}\biggr)\]
\item The magnetic field is assumed to lie in the $(x,z)$ plane, making an angle $\theta$
with the longitudinal axis:
\[ \vec{H} = H_0 (\sin \theta, 0, \cos \theta)\]
\end{enumerate}

%\section

\end{document}