\input{header.tex}
\usepackage[utf8]{inputenc}
\usepackage{amsmath}
\usepackage{amssymb}
\usepackage{amsfonts}
\usepackage{amssymb}
\usepackage{float}
\usepackage{indentfirst}
\usepackage{vmargin}
\usepackage{indentfirst}
\usepackage{titling}
\usepackage{color} 
\usepackage{siunitx}
\usepackage{xspace}
\usepackage{graphicx}
\usepackage{enumitem}
\usepackage{array}
\usepackage[backend=biber,backref=true,style=unsrt,
style=numeric-comp,block=ragged,firstinits=true]{biblatex}
\addbibresource{ref-notes.bib}
\bibliography{ref-notes}
\graphicspath{{plot_synthesis/} {Feynman/}}

\newcommand{\mastersig}{\ensuremath{\Im{\widehat{\Sigma}^{A,B}(k,E)}}\xspace}
\newcommand{\chiqw}{\ensuremath{\Im{\chi}(q,\omega)}\xspace}

\providecommand{\norm}[1]{\lVert#1\rVert}

\newcommand{\subtitle}[1]{%
  \posttitle{%
    \par\end{center}
    \begin{center}\large#1\end{center}
    \vskip0.5em}%
}

\title{F9800 Condensed Matter Physics II}
\subtitle{Final Exam}
%\author{}
\date{Spring 2014}

\begin{document}

\maketitle

\setlength{\unitlength}{1cm}
%\advance\textwidth by 3cm
%\advance\hoffset by -1.5cm 
\advance\textheight by 1cm
\advance\voffset by -1.5cm
\setmarginsrb{3cm}{0.5cm}{1.5cm}{1cm}{1cm}{1cm}{1cm}{1cm}
%\setlength{\parindent}{0cm}%

\pagestyle{plain}

\vspace{2cm}

\textbf{Date:} \hspace{6cm} \textbf{Name:}

\vspace{1cm}

\section{Band structure}

\subsection{Introduction}

Figure~\ref{fig:bands} shows the diagram of energy versus wave vector, for
electrons in a 1D solid.

\begin{figure}[h]
  \centering
  \includegraphics[width=10cm]{Final_bands.pdf}
  \caption{Band structure of a 1D solid.\label{fig:bands}}
\end{figure}

\subsection{Questions}

\begin{enumerate}[label=(\roman*)]
\item If we note $n$ the density of electrons, and $p$ the density of
  holes, what assertion can you formulate about the value of the ratio
  $\dfrac{p}{n}$?
\item Can you infer from Fig.~\ref{fig:bands} if this material has an
  even or odd number of electrons per unit cell? Justify your answer.
\item Which is greater: the effective masses of the electron, or that
  of the hole? Derive approximate expressions for the effective masses
  in terms of quantities in the diagram.
\end{enumerate}

\section{Carrier densities in a semiconducting material}

\subsection{Density of states in 3D}\label{section:DOS}

In this section, we establish a general expression for $g_n(E)$, the density of levels
per unit volume (further called "density of levels" or "density of
states") as a function of energy, for a given band $n$. The band dispersion is noted
$E_n(k)$.

\begin{enumerate}[label=(\roman*)]
\item Derive the general expression $$g_n(E) = \int\int_{S_n(E)} \dfrac{dS}{4
    \pi^3} \dfrac{1}{\abs{\nabla_k E_n(k)}}$$
\item Show that in the particular case of a system at zero
  temperature, with a parabolic band with effective mass $m^{*}$, and
  Fermi energy $E_F$, the density of electrons is: $$n_e =
  \dfrac{1}{3 \pi^2} \biggl[\dfrac{2 m^{*} (E_F-E_c)}{ \hbar^2} \biggr]^{3/2}$$
where $E_c$ is the energy at the bottom of the considered parabolic band.
\end{enumerate}

\subsection{Filling of levels in a semiconductor - $T=0$K}

In this section, we consider the direct band gap semiconductor GaAs,
with a band gap $E_g = 1.43$eV at 300K, and a second higher-lying
conduction band extremum at the $X$-point in the Brillouin zone,
located at $E_{gX} = 0.35$eV above the $\Gamma$ point conduction band
minimum (see Fig.~\ref{fig:SC_bands}). The effective masses are $m_e(\Gamma) =
0.065 m_0$ for the $\Gamma$-point electrons, $m_l(X) =
1.2 m_0$ and $m_t(X) = 0.3 m_0$ for the $X$-point electrons. All bands
are assumed to be parabolic.

In this section, we consider the system at
zero temperature, without doping, and assume we can externally shift
its Fermi level.

\begin{figure}[h]
  \centering
  \includegraphics[width=7cm]{SC_bands.pdf}
  \caption{Simplified band structure of GaAs.\label{fig:SC_bands}}
\end{figure}

The Brillouin zone, with the relevant symmetries and location of X
point, is illustrated in Fig~\ref{fig:SC_symmetry}

\begin{figure}[h]
  \centering
  \includegraphics[width=7cm]{SC_symmetry.pdf}
  \caption{Brillouin zone representation of GaAs.\label{fig:SC_symmetry}}
\end{figure}

\begin{enumerate}[label=(\roman*)]
\item What value does $E_F$ need to reach in order for the electron
  concentration at the $X$ point and the electron concentration at the
  $\Gamma$ point to be equal? What is then the total density of
  electrons in the conduction band? Do not forget to account for the
  symmetry of the system, and use $m^{*} = (m_l m_t^2)^{1/3}$ at the
  $X$-point.
\item If the $X$-point electron pocket is to stay empty, what is the
  highest possible electron density at the $\Gamma$-point?
  Compare such electron density with the electron density the
  $\Gamma$-point in the previous case.
\item What is the hole concentration in both situations considered above?
\end{enumerate}

\section{Donor states in Ge}

\subsection{Introduction}

The conduction band of Germanium exhibits four equivalent minima at
the L points of the Brillouin zone. The (orthonormal) electronic state
vectors in such directions are designated by the following symbols:

\begin{itemize}
\item $\ket{a}$ for the minimum in direction $[1\bar{1}\bar{1}]$
\item $\ket{b}$ for the minimum in direction $[\bar{1}1\bar{1}]$
\item $\ket{c}$ for the minimum in direction $[111]$
\item $\ket{d}$ for the minimum in direction $[\bar{1}\bar{1}1]$
\end{itemize}

\begin{figure}[h]
  \centering
  \includegraphics[width=7cm]{xyzcube.png}
  \caption{Regular tetrahedron with Ge atoms on the a, b, c,
    and d sites. A donor atom is located in the middle of the tetrahedron.\label{fig:tetra}}
\end{figure}

A donor atom is located at the center of a regular tetrahedron as
indicated in Fig~\ref{fig:tetra}. As a reminder, the group of
symmetries of the regular tetrahedraon is $T_d$, of order 24, with elements:

\begin{itemize}
\item $E$ (identity)
\item 8 rotations  $C_3$ about the diagonals of a cube.
\item 3 rotations $C_2$ about axes $x, y , z$.
\item 6 reflections $\sigma_d$ in planes containing one edge and the center of
  the tetrahedron (diagonal axes).
\item 6 improper rotations $S_4$ about axis $x, y , z$ (rotations
of angle $\pi/2$ followed by a reflection in a plane perpendicular to
the axis of rotation).
\end{itemize}

Its character table is the following:

\begin{center}
 \begin{tabular}{|| l | *{5}{c} ||} 
 \hline
   & E & 8$C_3$ & 3$C_2$ & 6$\sigma_d$ & 6$S_4$ \\ [0.5ex] 
 \hline\hline
 $A_1$ & 1 & 1 & 1 & 1 & 1 \\ 
 \hline
 $A_2$ & 1 & 1 & 1 & -1 & -1 \\ 
 \hline
 $E$ & 2 & -1 & 2 & 0 & 0 \\ 
 \hline
 $T_1$ & 3 & 0 & -1 & -1 & 1 \\ 
 \hline
 $T_2$ & 3 & 0 & -1 & 1 & -1 \\
 \hline
\end{tabular}
\end{center}

\subsection{Questions}

\begin{enumerate}[label=(\roman*)]
\item Apply one symmetry operation of each class of the $T_d$ group to
  the four dimensional vector $\begin{pmatrix}
\ket{a}\\ 
\ket{b}\\
\ket{c}\\
\ket{d}\\
\end{pmatrix}$
\item Using the previous result, establish the character table of the
  four-dimensional representation $R_4$ of the group $T_d$.
\item Using the character
  table of the $T_d$ group, decompose $R_4$ into its irreducible
  components. Such decomposition should be the result of an explicit
  calculation, i.e. the result should not only be established by inspection.
\item Verify that the following symmetrized combinations of state
  vectors are bases of the corresponding irreducible representations
  of the $T_d$ group:
  \begin {itemize}
  \item $A_1: \dfrac{\ket{a} + \ket{b} + \ket{c} + \ket{d}}{2}$
  \item $T_2: $ \begin{itemize}
    \item $\ket{pmpm} = \dfrac{\ket{a} - \ket{b} + \ket{c} -
        \ket{d}}{2}$
    \item $\ket{ppmm} = \dfrac{\ket{a} + \ket{b} - \ket{c} -
        \ket{d}}{2}$
    \item $\ket{pmmp} = \dfrac{\ket{a} - \ket{b} - \ket{c} +
        \ket{d}}{2}$
    \end{itemize}
  \end{itemize}
\end{enumerate}



%\section

\end{document}