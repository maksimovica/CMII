\input{header.tex}
\usepackage[utf8]{inputenc}
\usepackage{amsmath}
\usepackage{amssymb}
\usepackage{amsfonts}
\usepackage{amssymb}
\usepackage{float}
\usepackage{indentfirst}
\usepackage{vmargin}
\usepackage{indentfirst}
\usepackage{titling}
\usepackage{color} 
\usepackage{siunitx}
\usepackage{xspace}
\usepackage{graphicx}
\usepackage{enumitem}
\usepackage[backend=biber,backref=true,style=unsrt,
style=numeric-comp,block=ragged,firstinits=true]{biblatex}
\addbibresource{ref-notes.bib}
\bibliography{ref-notes}
\graphicspath{{plot_synthesis/} {Feynman/}}

\newcommand{\mastersig}{\ensuremath{\Im{\widehat{\Sigma}^{A,B}(k,E)}}\xspace}
\newcommand{\chiqw}{\ensuremath{\Im{\chi}(q,\omega)}\xspace}

\providecommand{\norm}[1]{\lVert#1\rVert}

\newcommand{\subtitle}[1]{%
  \posttitle{%
    \par\end{center}
    \begin{center}\large#1\end{center}
    \vskip0.5em}%
}


\title{Condensed Matter II}
\subtitle{Problem set \#4}
%\author{}
\date{Spring 2014}

\begin{document}

\maketitle

\setlength{\unitlength}{1cm}
%\advance\textwidth by 3cm
%\advance\hoffset by -1.5cm 
\advance\textheight by 1cm
\advance\voffset by -1.5cm
\setmarginsrb{3cm}{0.5cm}{1.5cm}{1cm}{1cm}{1cm}{1cm}{1cm}
%\setlength{\parindent}{0cm}%

\pagestyle{plain}

\section{Free electrons in Na}

\begin{figure}[h]
  \centering
  \includegraphics[width=14cm]{NA_bands.pdf}
  \caption{Energy bands in sodium along symmetry directions. Chang
    and Callaway, PRB 11, 1324 (1975) \label{fig:bands}}
\end{figure}

\begin{enumerate}[label=(\roman*)]
\item With the help of the computed band structure of Na in Fig.~\ref{fig:bands}, suggest what the structure of the empty BCC lattice must look
  like (assuming isotropic effective mass $m_0$).
\item How high above the bottom of the conduction band is the Fermi
  energy located in this model?
\item justify that Na be considered ``the simplest of the simple metals''.
\end{enumerate}

\section{Density of states of Si, Ge, Sn}

%\section

\end{document}