\input{header.tex}
\usepackage[utf8]{inputenc}
\usepackage{amsmath}
\usepackage{amssymb}
\usepackage{amsfonts}
\usepackage{amssymb}
\usepackage{float}
\usepackage{indentfirst}
\usepackage{vmargin}
\usepackage{indentfirst}
\usepackage{titling}
\usepackage{color} 
\usepackage{siunitx}
\usepackage{xspace}
\usepackage{graphicx}
\usepackage{enumitem}
\usepackage[backend=biber,backref=true,style=unsrt,
style=numeric-comp,block=ragged,firstinits=true]{biblatex}
\addbibresource{ref-notes.bib}
\bibliography{ref-notes}
\graphicspath{{plot_synthesis/} {Feynman/}}

\newcommand{\mastersig}{\ensuremath{\Im{\widehat{\Sigma}^{A,B}(k,E)}}\xspace}
\newcommand{\chiqw}{\ensuremath{\Im{\chi}(q,\omega)}\xspace}

\providecommand{\norm}[1]{\lVert#1\rVert}

\newcommand{\subtitle}[1]{%
  \posttitle{%
    \par\end{center}
    \begin{center}\large#1\end{center}
    \vskip0.5em}%
}


\title{Condensed Matter II}
\subtitle{Problem set \#3}
%\author{}
\date{March 2014}

\begin{document}

\maketitle

\setlength{\unitlength}{1cm}
%\advance\textwidth by 3cm
%\advance\hoffset by -1.5cm 
\advance\textheight by 1cm
\advance\voffset by -1.5cm
\setmarginsrb{3cm}{0.5cm}{1.5cm}{1cm}{1cm}{1cm}{1cm}{1cm}
%\setlength{\parindent}{0cm}%

\pagestyle{plain}

\section{$T_d$ group representation}

\subsection{Background}

The group of symmetry operations of the regular tetrahedron $T_d$ is
isomorphic to the group of permutations of four objects $P(4)$.

\begin{figure}[h]
  \centering
  \includegraphics[width=7cm]{xyzcube.png}
  \caption{Regular tetrahedron with vertices abcd.}
\end{figure}

The elements of the group $T_d$ are (Schoenfliss notation):

\begin{itemize}
\item $E$ (identity)
\item 8 rotations  $C_3$ about the diagonals of a cube.
\item 3 rotations $C_2$ about axes $x, y , z$.
\item 6 improper rotations $S_4$ about axis $x, y , z$ (rotations
of angle $\pi/2$ followed by a reflection in a plane perpendicular to
the axis of rotation).
\item 6 reflections $\sigma_d$ in planes containing one edge and the center of
  the tetrahedron.
\end{itemize}

The elements of the group $P(4)$ are:

\begin{itemize}
\item $E=$(abcd) $A=$(acbd) $B=$(cbad) $C=$(bacd) $D=$(cabd)
  $F=$(bcad) (perm. abc;d)
\item $G=$(abdc) $H=$(adbc) $J=$(dbac) $K=$(badc) $L=$(dabc)
  $M=$(bdac) (perm. abd;c)
\item $N=$(adcb) $O=$(acdb) $P=$(cdab) $Q=$(dacb) $R=$(cadb)
  $S=$(dcab) (perm. acd;b)
\item $T=$(dbca) $U=$(dcba) $V=$(cbda) $W=$(bdca) $X=$(cdba) $Y=$(bcda) (perm. bcd;a)
\end{itemize}

\subsection{Questions}

\begin{enumerate}[label=(\roman*)]
\item Show that $P(4)$ and $T_d$ are isomorphic.
\item Determine the periods of the elements of the group.
\item Determine at least 10 subgroups.
\item Is $\{ E, G, N, T \}$ a subgroup?
\item Determine the subgroups isomorphic to $P(3)$.
\item Partition the elements of $P(4)$ so that the elements of each
  subset have the same order (the order $n$ of element $X$ is the smallest $n \in
  \mathbb{N}$ such that $X^n=E$).
\item Determine the classes (sets of equivalent elements, through the
  relation $A \sim B \Leftrightarrow \exists X \in G, B = XAX^{-1}$)
\item Find several representations.
\item Determine the irreducible representations, and the character
  table of the group.
\end{enumerate}

\end{document}