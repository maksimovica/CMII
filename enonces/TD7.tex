\input{header.tex}
\usepackage[utf8]{inputenc}
\usepackage{amsmath}
\usepackage{amssymb}
\usepackage{amsfonts}
\usepackage{amssymb}
\usepackage{float}
\usepackage{indentfirst}
\usepackage{vmargin}
\usepackage{indentfirst}
\usepackage{titling}
\usepackage{color} 
\usepackage{siunitx}
\usepackage{xspace}
\usepackage{graphicx}
\usepackage{enumitem}
\usepackage[backend=biber,backref=true,style=unsrt,
style=numeric-comp,block=ragged,firstinits=true]{biblatex}
\addbibresource{ref-notes.bib}
\bibliography{ref-notes}
\graphicspath{{plot_synthesis/} {Feynman/}}

\newcommand{\mastersig}{\ensuremath{\Im{\widehat{\Sigma}^{A,B}(k,E)}}\xspace}
\newcommand{\chiqw}{\ensuremath{\Im{\chi}(q,\omega)}\xspace}

\providecommand{\norm}[1]{\lVert#1\rVert}

\newcommand{\subtitle}[1]{%
  \posttitle{%
    \par\end{center}
    \begin{center}\large#1\end{center}
    \vskip0.5em}%
}

\title{Condensed Matter II}
\subtitle{Problem set \#7}
%\author{}
\date{Spring 2014}

\begin{document}

\maketitle

\setlength{\unitlength}{1cm}
%\advance\textwidth by 3cm
%\advance\hoffset by -1.5cm 
\advance\textheight by 1cm
\advance\voffset by -1.5cm
\setmarginsrb{3cm}{0.5cm}{1.5cm}{1cm}{1cm}{1cm}{1cm}{1cm}
%\setlength{\parindent}{0cm}%

\pagestyle{plain}

\section{Donor states in Si}

\subsection{Background}

The conduction band of Silicium exhibits six equivalent minima in the
direction $\Delta$ in the first Brillouin zone. The electronic state
vectors in such directions are designated by the following symbols:

\begin{itemize}
\item $\ket{x}$ for the minimum in direction $[100]$
\item $\ket{y}$ for the minimum in direction $[010]$
\item $\ket{z}$ for the minimum in direction $[001]$
\item $\ket{\bar{x}}$ for the minimum in direction $[\bar{1}00]$
\item $\ket{\bar{y}}$ for the minimum in direction $[0\bar{1}0]$
\item $\ket{\bar{z}}$ for the minimum in direction $[00\bar{1}]$
\end{itemize}

\begin{figure}[h]
  \centering
  \includegraphics[width=7cm]{xyzcube.png}
  \caption{Regular tetrahedron with vertices Si atoms on the a, b, c,
    d sites. In the middle of the tetrahedron is a donor atom.\label{fig:tetra}}
\end{figure}

A donor atom is located at the center of a regular tetrahedron as
indicated in Fig~\ref{fig:tetra}. As a reminder, the symmetry of the
tetrahedraon is $T_d$, of order 24, with elements:

\begin{itemize}
\item $E$ (identity)
\item 8 rotations  $C_3$ about the diagonals of a cube.
\item 3 rotations $C_2$ about axes $x, y , z$.
\item 6 improper rotations $S_4$ about axis $x, y , z$ (rotations
of angle $\pi/2$ followed by a reflection in a plane perpendicular to
the axis of rotation).
\item 6 reflections $\sigma_d$ in planes containing one edge and the center of
  the tetrahedron.
\end{itemize}

\subsection{Questions}

\begin{enumerate}[label=(\roman*)]
\item Apply a symmetry operation of each class of the $T_d$ group to the six
  dimensional vector $\begin{pmatrix}
\ket{x}\\ 
\ket{y}\\
\ket{z}\\
\ket{\bar{x}}\\
\ket{\bar{y}}\\
\ket{\bar{z}}
\end{pmatrix}$
\item Use the previous result to establish the character table of the
  six-dimensional representation $R_6$ of the group $T_d$.
\item Using the previously established (Cf Problem Set \#3) character
  table of the $T_d$ group, decompose $R_6$ into its irreducible
  components.
\item Verify that the following vector states are bases of the
  corresponding irreducible representations:
  \begin {itemize}
  \item $A_1: \dfrac{\ket{x} + \ket{y} + \ket{z} + \ket{\bar{x}} +
      \ket{\bar{y}} + \ket{\bar{z}}}{\sqrt{6}} $
  \item $E: \dfrac{\ket{x} - \ket{y} + \ket{\bar{x}} -
      \ket{\bar{y}}}{2}$, $\dfrac{\ket{x} + \ket{y} -2 \ket{z} + \ket{\bar{x}} +
      \ket{\bar{y}} -2 \ket{\bar{z}}}{\sqrt{12}}$
  \item $T_2: \dfrac{\ket{x} - \ket{\bar{x}}}{\sqrt{2}}$,
    $\dfrac{\ket{y} - \ket{\bar{y}}}{\sqrt{2}}$, $\dfrac{\ket{z} - \ket{\bar{z}}}{\sqrt{2}}$
  \end{itemize}
\item What is the physical significance of the findings of the
  previous question?
\end{enumerate}



%\section

\end{document}