
% Default to the notebook output style

    


% Inherit from the specified cell style.




    
\documentclass{article}

    
    
    \usepackage{graphicx} % Used to insert images
    \usepackage{adjustbox} % Used to constrain images to a maximum size 
    \usepackage{color} % Allow colors to be defined
    \usepackage{enumerate} % Needed for markdown enumerations to work
    \usepackage{geometry} % Used to adjust the document margins
    \usepackage{amsmath} % Equations
    \usepackage{amssymb} % Equations
    \usepackage[mathletters]{ucs} % Extended unicode (utf-8) support
    \usepackage[utf8x]{inputenc} % Allow utf-8 characters in the tex document
    \usepackage{fancyvrb} % verbatim replacement that allows latex
    \usepackage{grffile} % extends the file name processing of package graphics 
                         % to support a larger range 
    % The hyperref package gives us a pdf with properly built
    % internal navigation ('pdf bookmarks' for the table of contents,
    % internal cross-reference links, web links for URLs, etc.)
    \usepackage{hyperref}
    \usepackage{longtable} % longtable support required by pandoc >1.10
    

    
    
    \definecolor{orange}{cmyk}{0,0.4,0.8,0.2}
    \definecolor{darkorange}{rgb}{.71,0.21,0.01}
    \definecolor{darkgreen}{rgb}{.12,.54,.11}
    \definecolor{myteal}{rgb}{.26, .44, .56}
    \definecolor{gray}{gray}{0.45}
    \definecolor{lightgray}{gray}{.95}
    \definecolor{mediumgray}{gray}{.8}
    \definecolor{inputbackground}{rgb}{.95, .95, .85}
    \definecolor{outputbackground}{rgb}{.95, .95, .95}
    \definecolor{traceback}{rgb}{1, .95, .95}
    % ansi colors
    \definecolor{red}{rgb}{.6,0,0}
    \definecolor{green}{rgb}{0,.65,0}
    \definecolor{brown}{rgb}{0.6,0.6,0}
    \definecolor{blue}{rgb}{0,.145,.698}
    \definecolor{purple}{rgb}{.698,.145,.698}
    \definecolor{cyan}{rgb}{0,.698,.698}
    \definecolor{lightgray}{gray}{0.5}
    
    % bright ansi colors
    \definecolor{darkgray}{gray}{0.25}
    \definecolor{lightred}{rgb}{1.0,0.39,0.28}
    \definecolor{lightgreen}{rgb}{0.48,0.99,0.0}
    \definecolor{lightblue}{rgb}{0.53,0.81,0.92}
    \definecolor{lightpurple}{rgb}{0.87,0.63,0.87}
    \definecolor{lightcyan}{rgb}{0.5,1.0,0.83}
    
    % commands and environments needed by pandoc snippets
    % extracted from the output of `pandoc -s`
    \DefineVerbatimEnvironment{Highlighting}{Verbatim}{commandchars=\\\{\}}
    % Add ',fontsize=\small' for more characters per line
    \newenvironment{Shaded}{}{}
    \newcommand{\KeywordTok}[1]{\textcolor[rgb]{0.00,0.44,0.13}{\textbf{{#1}}}}
    \newcommand{\DataTypeTok}[1]{\textcolor[rgb]{0.56,0.13,0.00}{{#1}}}
    \newcommand{\DecValTok}[1]{\textcolor[rgb]{0.25,0.63,0.44}{{#1}}}
    \newcommand{\BaseNTok}[1]{\textcolor[rgb]{0.25,0.63,0.44}{{#1}}}
    \newcommand{\FloatTok}[1]{\textcolor[rgb]{0.25,0.63,0.44}{{#1}}}
    \newcommand{\CharTok}[1]{\textcolor[rgb]{0.25,0.44,0.63}{{#1}}}
    \newcommand{\StringTok}[1]{\textcolor[rgb]{0.25,0.44,0.63}{{#1}}}
    \newcommand{\CommentTok}[1]{\textcolor[rgb]{0.38,0.63,0.69}{\textit{{#1}}}}
    \newcommand{\OtherTok}[1]{\textcolor[rgb]{0.00,0.44,0.13}{{#1}}}
    \newcommand{\AlertTok}[1]{\textcolor[rgb]{1.00,0.00,0.00}{\textbf{{#1}}}}
    \newcommand{\FunctionTok}[1]{\textcolor[rgb]{0.02,0.16,0.49}{{#1}}}
    \newcommand{\RegionMarkerTok}[1]{{#1}}
    \newcommand{\ErrorTok}[1]{\textcolor[rgb]{1.00,0.00,0.00}{\textbf{{#1}}}}
    \newcommand{\NormalTok}[1]{{#1}}
    
    % Define a nice break command that doesn't care if a line doesn't already
    % exist.
    \def\br{\hspace*{\fill} \\* }
    % Math Jax compatability definitions
    \def\gt{>}
    \def\lt{<}
    % Document parameters
    \title{Free\_electrons\_in\_Na}
    
    
    

    % Pygments definitions
    
\makeatletter
\def\PY@reset{\let\PY@it=\relax \let\PY@bf=\relax%
    \let\PY@ul=\relax \let\PY@tc=\relax%
    \let\PY@bc=\relax \let\PY@ff=\relax}
\def\PY@tok#1{\csname PY@tok@#1\endcsname}
\def\PY@toks#1+{\ifx\relax#1\empty\else%
    \PY@tok{#1}\expandafter\PY@toks\fi}
\def\PY@do#1{\PY@bc{\PY@tc{\PY@ul{%
    \PY@it{\PY@bf{\PY@ff{#1}}}}}}}
\def\PY#1#2{\PY@reset\PY@toks#1+\relax+\PY@do{#2}}

\expandafter\def\csname PY@tok@gd\endcsname{\def\PY@tc##1{\textcolor[rgb]{0.63,0.00,0.00}{##1}}}
\expandafter\def\csname PY@tok@gu\endcsname{\let\PY@bf=\textbf\def\PY@tc##1{\textcolor[rgb]{0.50,0.00,0.50}{##1}}}
\expandafter\def\csname PY@tok@gt\endcsname{\def\PY@tc##1{\textcolor[rgb]{0.00,0.27,0.87}{##1}}}
\expandafter\def\csname PY@tok@gs\endcsname{\let\PY@bf=\textbf}
\expandafter\def\csname PY@tok@gr\endcsname{\def\PY@tc##1{\textcolor[rgb]{1.00,0.00,0.00}{##1}}}
\expandafter\def\csname PY@tok@cm\endcsname{\let\PY@it=\textit\def\PY@tc##1{\textcolor[rgb]{0.25,0.50,0.50}{##1}}}
\expandafter\def\csname PY@tok@vg\endcsname{\def\PY@tc##1{\textcolor[rgb]{0.10,0.09,0.49}{##1}}}
\expandafter\def\csname PY@tok@m\endcsname{\def\PY@tc##1{\textcolor[rgb]{0.40,0.40,0.40}{##1}}}
\expandafter\def\csname PY@tok@mh\endcsname{\def\PY@tc##1{\textcolor[rgb]{0.40,0.40,0.40}{##1}}}
\expandafter\def\csname PY@tok@go\endcsname{\def\PY@tc##1{\textcolor[rgb]{0.53,0.53,0.53}{##1}}}
\expandafter\def\csname PY@tok@ge\endcsname{\let\PY@it=\textit}
\expandafter\def\csname PY@tok@vc\endcsname{\def\PY@tc##1{\textcolor[rgb]{0.10,0.09,0.49}{##1}}}
\expandafter\def\csname PY@tok@il\endcsname{\def\PY@tc##1{\textcolor[rgb]{0.40,0.40,0.40}{##1}}}
\expandafter\def\csname PY@tok@cs\endcsname{\let\PY@it=\textit\def\PY@tc##1{\textcolor[rgb]{0.25,0.50,0.50}{##1}}}
\expandafter\def\csname PY@tok@cp\endcsname{\def\PY@tc##1{\textcolor[rgb]{0.74,0.48,0.00}{##1}}}
\expandafter\def\csname PY@tok@gi\endcsname{\def\PY@tc##1{\textcolor[rgb]{0.00,0.63,0.00}{##1}}}
\expandafter\def\csname PY@tok@gh\endcsname{\let\PY@bf=\textbf\def\PY@tc##1{\textcolor[rgb]{0.00,0.00,0.50}{##1}}}
\expandafter\def\csname PY@tok@ni\endcsname{\let\PY@bf=\textbf\def\PY@tc##1{\textcolor[rgb]{0.60,0.60,0.60}{##1}}}
\expandafter\def\csname PY@tok@nl\endcsname{\def\PY@tc##1{\textcolor[rgb]{0.63,0.63,0.00}{##1}}}
\expandafter\def\csname PY@tok@nn\endcsname{\let\PY@bf=\textbf\def\PY@tc##1{\textcolor[rgb]{0.00,0.00,1.00}{##1}}}
\expandafter\def\csname PY@tok@no\endcsname{\def\PY@tc##1{\textcolor[rgb]{0.53,0.00,0.00}{##1}}}
\expandafter\def\csname PY@tok@na\endcsname{\def\PY@tc##1{\textcolor[rgb]{0.49,0.56,0.16}{##1}}}
\expandafter\def\csname PY@tok@nb\endcsname{\def\PY@tc##1{\textcolor[rgb]{0.00,0.50,0.00}{##1}}}
\expandafter\def\csname PY@tok@nc\endcsname{\let\PY@bf=\textbf\def\PY@tc##1{\textcolor[rgb]{0.00,0.00,1.00}{##1}}}
\expandafter\def\csname PY@tok@nd\endcsname{\def\PY@tc##1{\textcolor[rgb]{0.67,0.13,1.00}{##1}}}
\expandafter\def\csname PY@tok@ne\endcsname{\let\PY@bf=\textbf\def\PY@tc##1{\textcolor[rgb]{0.82,0.25,0.23}{##1}}}
\expandafter\def\csname PY@tok@nf\endcsname{\def\PY@tc##1{\textcolor[rgb]{0.00,0.00,1.00}{##1}}}
\expandafter\def\csname PY@tok@si\endcsname{\let\PY@bf=\textbf\def\PY@tc##1{\textcolor[rgb]{0.73,0.40,0.53}{##1}}}
\expandafter\def\csname PY@tok@s2\endcsname{\def\PY@tc##1{\textcolor[rgb]{0.73,0.13,0.13}{##1}}}
\expandafter\def\csname PY@tok@vi\endcsname{\def\PY@tc##1{\textcolor[rgb]{0.10,0.09,0.49}{##1}}}
\expandafter\def\csname PY@tok@nt\endcsname{\let\PY@bf=\textbf\def\PY@tc##1{\textcolor[rgb]{0.00,0.50,0.00}{##1}}}
\expandafter\def\csname PY@tok@nv\endcsname{\def\PY@tc##1{\textcolor[rgb]{0.10,0.09,0.49}{##1}}}
\expandafter\def\csname PY@tok@s1\endcsname{\def\PY@tc##1{\textcolor[rgb]{0.73,0.13,0.13}{##1}}}
\expandafter\def\csname PY@tok@sh\endcsname{\def\PY@tc##1{\textcolor[rgb]{0.73,0.13,0.13}{##1}}}
\expandafter\def\csname PY@tok@sc\endcsname{\def\PY@tc##1{\textcolor[rgb]{0.73,0.13,0.13}{##1}}}
\expandafter\def\csname PY@tok@sx\endcsname{\def\PY@tc##1{\textcolor[rgb]{0.00,0.50,0.00}{##1}}}
\expandafter\def\csname PY@tok@bp\endcsname{\def\PY@tc##1{\textcolor[rgb]{0.00,0.50,0.00}{##1}}}
\expandafter\def\csname PY@tok@c1\endcsname{\let\PY@it=\textit\def\PY@tc##1{\textcolor[rgb]{0.25,0.50,0.50}{##1}}}
\expandafter\def\csname PY@tok@kc\endcsname{\let\PY@bf=\textbf\def\PY@tc##1{\textcolor[rgb]{0.00,0.50,0.00}{##1}}}
\expandafter\def\csname PY@tok@c\endcsname{\let\PY@it=\textit\def\PY@tc##1{\textcolor[rgb]{0.25,0.50,0.50}{##1}}}
\expandafter\def\csname PY@tok@mf\endcsname{\def\PY@tc##1{\textcolor[rgb]{0.40,0.40,0.40}{##1}}}
\expandafter\def\csname PY@tok@err\endcsname{\def\PY@bc##1{\setlength{\fboxsep}{0pt}\fcolorbox[rgb]{1.00,0.00,0.00}{1,1,1}{\strut ##1}}}
\expandafter\def\csname PY@tok@kd\endcsname{\let\PY@bf=\textbf\def\PY@tc##1{\textcolor[rgb]{0.00,0.50,0.00}{##1}}}
\expandafter\def\csname PY@tok@ss\endcsname{\def\PY@tc##1{\textcolor[rgb]{0.10,0.09,0.49}{##1}}}
\expandafter\def\csname PY@tok@sr\endcsname{\def\PY@tc##1{\textcolor[rgb]{0.73,0.40,0.53}{##1}}}
\expandafter\def\csname PY@tok@mo\endcsname{\def\PY@tc##1{\textcolor[rgb]{0.40,0.40,0.40}{##1}}}
\expandafter\def\csname PY@tok@kn\endcsname{\let\PY@bf=\textbf\def\PY@tc##1{\textcolor[rgb]{0.00,0.50,0.00}{##1}}}
\expandafter\def\csname PY@tok@mi\endcsname{\def\PY@tc##1{\textcolor[rgb]{0.40,0.40,0.40}{##1}}}
\expandafter\def\csname PY@tok@gp\endcsname{\let\PY@bf=\textbf\def\PY@tc##1{\textcolor[rgb]{0.00,0.00,0.50}{##1}}}
\expandafter\def\csname PY@tok@o\endcsname{\def\PY@tc##1{\textcolor[rgb]{0.40,0.40,0.40}{##1}}}
\expandafter\def\csname PY@tok@kr\endcsname{\let\PY@bf=\textbf\def\PY@tc##1{\textcolor[rgb]{0.00,0.50,0.00}{##1}}}
\expandafter\def\csname PY@tok@s\endcsname{\def\PY@tc##1{\textcolor[rgb]{0.73,0.13,0.13}{##1}}}
\expandafter\def\csname PY@tok@kp\endcsname{\def\PY@tc##1{\textcolor[rgb]{0.00,0.50,0.00}{##1}}}
\expandafter\def\csname PY@tok@w\endcsname{\def\PY@tc##1{\textcolor[rgb]{0.73,0.73,0.73}{##1}}}
\expandafter\def\csname PY@tok@kt\endcsname{\def\PY@tc##1{\textcolor[rgb]{0.69,0.00,0.25}{##1}}}
\expandafter\def\csname PY@tok@ow\endcsname{\let\PY@bf=\textbf\def\PY@tc##1{\textcolor[rgb]{0.67,0.13,1.00}{##1}}}
\expandafter\def\csname PY@tok@sb\endcsname{\def\PY@tc##1{\textcolor[rgb]{0.73,0.13,0.13}{##1}}}
\expandafter\def\csname PY@tok@k\endcsname{\let\PY@bf=\textbf\def\PY@tc##1{\textcolor[rgb]{0.00,0.50,0.00}{##1}}}
\expandafter\def\csname PY@tok@se\endcsname{\let\PY@bf=\textbf\def\PY@tc##1{\textcolor[rgb]{0.73,0.40,0.13}{##1}}}
\expandafter\def\csname PY@tok@sd\endcsname{\let\PY@it=\textit\def\PY@tc##1{\textcolor[rgb]{0.73,0.13,0.13}{##1}}}

\def\PYZbs{\char`\\}
\def\PYZus{\char`\_}
\def\PYZob{\char`\{}
\def\PYZcb{\char`\}}
\def\PYZca{\char`\^}
\def\PYZam{\char`\&}
\def\PYZlt{\char`\<}
\def\PYZgt{\char`\>}
\def\PYZsh{\char`\#}
\def\PYZpc{\char`\%}
\def\PYZdl{\char`\$}
\def\PYZhy{\char`\-}
\def\PYZsq{\char`\'}
\def\PYZdq{\char`\"}
\def\PYZti{\char`\~}
% for compatibility with earlier versions
\def\PYZat{@}
\def\PYZlb{[}
\def\PYZrb{]}
\makeatother


    % Exact colors from NB
    \definecolor{incolor}{rgb}{0.0, 0.0, 0.5}
    \definecolor{outcolor}{rgb}{0.545, 0.0, 0.0}



    
    % Prevent overflowing lines due to hard-to-break entities
    \sloppy 
    % Setup hyperref package
    \hypersetup{
      breaklinks=true,  % so long urls are correctly broken across lines
      colorlinks=true,
      urlcolor=blue,
      linkcolor=darkorange,
      citecolor=darkgreen,
      }
    % Slightly bigger margins than the latex defaults
    
    \geometry{verbose,tmargin=1in,bmargin=1in,lmargin=1in,rmargin=1in}
    
    

    \begin{document}
    
    
    \maketitle
    
    

    
    The following document is available for
consultation/download/modification at

http://nbviewer.ipython.org/github/dombrno/CMII/blob/master/notebooks/Free\_electrons\_in\_Na.ipynb?create=1


    \section{Qualitative study}



    \subsection{\emph{With the help of the computed band structure of Na, suggest what
the structure of the empty BCC lattice must look like (assuming
isotropic effective mass $m_0$)}}


    The empty BCC lattice is obtained when setting $V \equiv 0$. The effect
of the weak finite potential $V$ is to lift the accidental band
degeneracies at $k=0$ and at the zone boundaries (high symmetry points)
in the Brillouin zone. Such splittings vary for the various bands and at
different high symmetry points. In order to obtain the empty lattice
structure from the real structure, one needs to merge the bands at the
places where the effect of the finite value of $V$ is to split them. The
final result remains close to the computed band structure, because the
nearly-free electron approximation is quite justified in the case of Na.


    \subsection{\emph{How high above the bottom of the conduction band is the Fermi
energy located in this model?}}


    For the alkali metals, the nearly-free electron approximation is valid.
This leads to a nearly spherical Fermi surface, and small band gaps. The
crystal structure for the alkali metals is body centered cubic (BCC).

In the figure representing the computed band structure of Na, the lowest
band is the 3s conduction band. The filled valence bands lie much lower
in energy and are not represented. For the case of Na, the 3s conduction
band is very nearly free electron--like and the dispersion relations are
closely isotropic. The isotropic effective mass model with parameter
$m_0$ is therefore a very good approximation.

Since Na has a the {[}Ne{]}$3s^1$ electronic structure, the Fermi level
is determined so that the 3s band is exactly half--occupied. Therefore
the radius of the Fermi surface $k_F$ satisfies the relation
\[\dfrac{4 \pi k_F^3}{3}=\dfrac{1}{2}*2*\left(\dfrac{2\pi}{a}\right)^3 \Rightarrow k_F = \left(\dfrac{3}{4\pi}\right)^{1/3}\left(\dfrac{2\pi}{a}\right) \simeq 0.62 \left(\dfrac{2\pi}{a}\right)\]

The shortest distance from $\Gamma$ to a Brillouin zone face is
$\Gamma \text{N} = \left(\dfrac{2\pi}{a}\right) \sqrt{(1/2)^2+(1/2)^2+0^2} \simeq 0.71 \left(\dfrac{2\pi}{a}\right)$

This indicates that the Fermi sphere remains a safe distance away from
the zone boundary, where only narrow band gaps distort the otherwise
quasi free-electron like dispersion. It is therefore quite a safe
approximation to consider a quadratic dispersion with effective mass
$m_0$ for all electrons of Na, and a spherical Fermi surface. This is
also visible in the position of the $E_F$ line in the computed band
structure, which lies safely below the lowest lying band gap at the N
point.

The energy $E_F$ thus verifies
$E_F \simeq \dfrac{\hbar^2 k_F^2}{2m_0} = \dfrac{\hbar^2(3/4\pi)^{2/3}}{2m_0}\left(\dfrac{2\pi}{a}\right)^2$

Using:

\begin{enumerate}
\def\labelenumi{\arabic{enumi}.}
\itemsep1pt\parskip0pt\parsep0pt
\item
  $a=0.423 \times 10^{-9}$m
\item
  $m_0 = 0.97 \cdot m_e = 0.97 \cdot 9.1 \times 10^{-31}$kg
\item
  $\hbar = 1.05 \times 10^{-34}\text{m}^2$.kg/s
\end{enumerate}

we get
$E_F \simeq 5.2 \times 10^{-19}\text{J} = 3.2 \text{eV} = 0.24 \text{Ry}$

Ref:

\begin{enumerate}
\def\labelenumi{\arabic{enumi}.}
\itemsep1pt\parskip0pt\parsep0pt
\item
  Ashcroft Mermin, Ch. 15
\item
  Pastori-Paravicini, p.~223 for effective masses.
\end{enumerate}


    \subsection{\emph{Justify that Na may be considered ``the simplest of the simple
metals''}}


    As illustrated above, the various effects arising from the periodic
potential of the lattice materialize in the most simple manner in Na:

\begin{enumerate}
\def\labelenumi{\arabic{enumi}.}
\itemsep1pt\parskip0pt\parsep0pt
\item
  The effective mass is isotropic
\item
  The effective mass is very close the the rest mass of a free electron
\item
  The Fermi sphere does not get close to the band gaps.
\item
  The width of the band gaps is small compared to bandwidth
\item
  For all practical purposes, the dispersion relation of the electrons
  is therefore identical to the free electron dispersion relation
\end{enumerate}

In other terms, it is not possible to find a real material which
deviates less from the free electron gas case, than does Na. As a
consequence, it is the most simple to model.


    \section{\emph{Quantitative study}}



    \subsection{General expression of energy}


    For a 3D crystal, the free electron energies and wave functions can be
expressed in the following way:

Decompose the plane wave vector as a sum of Bloch wave vector and
reciprocal lattice vector, where the Bloch wave vector $k$ is restricted
to the first Brillouin zone. The reciprocal lattice vectors are given by
$K_l =l_1 b_1 + l_2 b_2 + l_2 b_3$ where $(l_1, l_2, l_3)$ are integers
and $b_i$ are primitive translations of the reciprocal lattice. The
plane wave function is thus given by
$\Psi_l(k, r)=e^{ik\cdot r}e^{iK_l\cdot r}$, where the second factor has
the periodicity of the lattice since $e^{iK_l\cdot R} = 1$ for any
vector $R$ of the direct lattice, i.e.~this form respects Bloch's
theorem.

The energy of such wave function is then given by
$E_l(k)=\dfrac{\hbar^2}{2m}(k+K_l)^2$. It follows that each band is
labelled by $l \equiv (l_1, l_2, l_3)$, with the corresponding wave
function and energy expression above.


    \subsubsection{The Body Centered Cubic lattice}


    The primitive translations of the reciprocal lattice are:
$b_1 = \dfrac{2\pi}{a}(x+z)$, $b_2 = \dfrac{2\pi}{a}(-x+y)$,
$b_3 = \dfrac{2\pi}{a}(-y+z)$.

A general reciprocal lattice vector is thus given by
$G_{h_1 h_2 h_3 } = \dfrac{2\pi}{a}[(h_1-h_2)x + (h_2-h_3)y + (h_1+h_3)z]$.

The 12 shortest reciprocal lattice vectors are
$\pm \dfrac{2\pi}{a}(x \pm y)$, $\pm \dfrac{2\pi}{a}(y \pm z)$,
$\pm \dfrac{2\pi}{a}(x \pm z)$

The empty lattice energy bands can be written as
$E_{l_1 l_2 l_3}= \dfrac{2\pi^2 \hbar^2}{ma^2}[(l_1 + \xi)^2+(l_2 + \eta)^2+(l_3 + \zeta)^2]$
with $l_1=h_1-h_2$, $l_2=h_2-h_3$, $l_3 = h_1+h_3$ and $h_i$ an integer,
and $k=\dfrac{2\pi}{a}(\xi, \eta, \zeta)$ in the first Brillouin zone.

The special points have the following coordinates:

$H= \dfrac{2\pi}{a} (1, 0, 0)$, $P= \dfrac{2\pi}{a} (1/2, 1/2, 1/2)$,
$N= \dfrac{2\pi}{a} (1/2, 1/2, 0)$

With this, the energy levels at the special points may be written as:

$E_{h_1 h_2 h_2}(\Gamma) = (h_1 - h_2)^2 + (h_2-h_3)^2 + (h_1+h_3)^2$

$E_{h_1 h_2 h_2}(H) = (h_1 - h_2 + 1)^2 + (h_2-h_3)^2 + (h_1+h_3)^2$

$E_{h_1 h_2 h_2}(P) = (h_1 - h_2 + \dfrac{1}{2})^2 + (h_2-h_3+\dfrac{1}{2})^2 + (h_1+h_3+\dfrac{1}{2})^2$

$E_{h_1 h_2 h_2}(N) = (h_1 - h_2 + \dfrac{1}{2})^2 + (h_2-h_3+\dfrac{1}{2})^2 + (h_1+h_3)^2$

These expressions allow us to determine the authorized values of the
energy levels at all special points, as well as determine their
degeneracy. Such points should then be joined by parabolic branches in
order to build the empty lattice band structure. For a detailed look at
the result, please check \textbf{Dresselhaus, Fig.12.6.a., p293}.

Refs:

\textbf{Dresselhaus, Chapter 12}

\textbf{Pengcheng Dai, Lecture notes, Physics 671, University of
Tennessee, Knoxville}

    \begin{Verbatim}[commandchars=\\\{\}]
{\color{incolor}In [{\color{incolor}3}]:} \PY{k+kn}{from} \PY{n+nn}{IPython.display} \PY{k+kn}{import} \PY{n}{Image}
        \PY{n}{i} \PY{o}{=} \PY{n}{Image}\PY{p}{(}\PY{n}{filename}\PY{o}{=}\PY{l+s}{\PYZsq{}}\PY{l+s}{empty\PYZus{}BCC\PYZus{}bands.png}\PY{l+s}{\PYZsq{}}\PY{p}{)}
        \PY{n}{i}
\end{Verbatim}
\texttt{\color{outcolor}Out[{\color{outcolor}3}]:}
    
    \begin{center}
    \adjustimage{max size={0.9\linewidth}{0.9\paperheight}}{Free_electrons_in_Na_files/Free_electrons_in_Na_13_0.png}
    \end{center}
    { \hspace*{\fill} \\}
    


    \subsubsection{Group of a vector $k$}


    The group of symmetry operations which leave the lattice invariant also
leaves the reciprocal lattice invariant. Suppose we know some wave
function $\Psi_k$ . A rotation or reflection operation of the point
group acting on $\Psi_k$ will give the same result as the rotation or
reflection of $k$ itself, that is
$R\Psi_k(x) = \Psi_{Rk}(x) = \Psi_k(R^{−1}x)$. Here we used the fact
that applying the same orthogonal transformation to both vectors in a
scalar product does not change the value of the product, for example
$k \cdot R^{−1}r = Rk \cdot RR^{−1}r = Rk \cdot r$. By applying every
symmetry of the group to a wave vector k, we generate the star of k. For
the BCC lattice, all operations leave $\Gamma$ invariant so $\Gamma$ is
its own star. For a general point k, there will be $g$ (where $g$ is the
order of the group of the crystal) points in the star of k.

The group (or subgroup of the original point group) of rotations and
reflections that transform $k$ into itself or into a new $k$ vector
separated from the original $k$ point by a reciprocal lattice vector,
are said to belong to the group of the wave vector $k$.

When the empty lattice bands are calculated, there are often a number of
degenerate bands at points of high symmetry like the $\Gamma$ point, or
the X point. Because the operations of the group of the wave vector
$\Gamma$ (or X) leave this point invariant, one can construct linear
combinations of these degenerate states that belong to representations
of the group of the wave vector $\Gamma$ (or X etc.). Once these linear
combinations are built and the corresponding representation identified,
one can infer what degeneracies will be lifted by the introduction of a
finite crystal potential.


    \subsubsection{Lifting degeneracies: finite potential}


    \textbf{Th. on matrix elements}: Matrix elements of any operator which
is invariant under all the operations of a group are zero between
functions belonging to different IR's of the group. Matrix elements are
also zero between functions belonging to different rows of the same
representation.

As a consequence, if the degenerate states are classified according to
the IR's of the group of the wave vector, many of the degenerate states
will belong to different IR's and therefore the off-diagonal matrix
elements of $V(r)$ between them will vanish. The matrix of $V(r)$ will
acquire a block form, with the dimension of each block giving the new
degeneracy level of the split energy levels. This lets one give a
praticularly simple form to the secular equation, which allows an easier
computation of the energy levels.

The IR to which a direction of the crystal belongs also determines which
lines can or cannot cross in the band structure profile: If they belong
to the same IR, the are coupled by the potential, and therefore cannot
cross. If they belong to different IRs, they may cross. Note such an
interesting occurence in \textbf{Fig. 12.6. of Dresselhaus, 293}, where
we clearly see the two $F1$ lines repelling each other, while $F_1$ and
$F_3$ have no problem crossing each other.

Following the previous calculation, it is in this way possible to
identify to which IR each wave function belongs, and thus to name each
point and branch, as is done in the complete computed band structure by
Ching and Callaway.


    \subsection{Result of analytical computation of bands:}


    $E_{h_1 h_2 h_2}(\Gamma) = (h_1 - h_2)^2 + (h_2-h_3)^2 + (h_1+h_3)^2$

$E_{h_1 h_2 h_2}(H) = (h_1 - h_2 + 1)^2 + (h_2-h_3)^2 + (h_1+h_3)^2$

$E_{h_1 h_2 h_2}(P) = (h_1 - h_2 + \dfrac{1}{2})^2 + (h_2-h_3+\dfrac{1}{2})^2 + (h_1+h_3+\dfrac{1}{2})^2$

$E_{h_1 h_2 h_2}(N) = (h_1 - h_2 + \dfrac{1}{2})^2 + (h_2-h_3+\dfrac{1}{2})^2 + (h_1+h_3)^2$

    \begin{Verbatim}[commandchars=\\\{\}]
{\color{incolor}In [{\color{incolor}47}]:} \PY{o}{\PYZpc{}}\PY{k}{matplotlib} \PY{n}{inline}
         \PY{k+kn}{import} \PY{n+nn}{matplotlib.pyplot} \PY{k+kn}{as} \PY{n+nn}{plt}
         \PY{c}{\PYZsh{}from pylab import figure, xticks, yticks}
         \PY{c}{\PYZsh{}import pylab}
         \PY{k+kn}{from} \PY{n+nn}{IPython.display} \PY{k+kn}{import} \PY{n}{display}\PY{p}{,} \PY{n}{Math}\PY{p}{,} \PY{n}{Latex}
         
         \PY{n}{special\PYZus{}points} \PY{o}{=} \PY{p}{[}\PY{l+s}{\PYZdq{}}\PY{l+s}{G}\PY{l+s}{\PYZdq{}}\PY{p}{,} \PY{l+s}{\PYZdq{}}\PY{l+s}{H}\PY{l+s}{\PYZdq{}}\PY{p}{,} \PY{l+s}{\PYZdq{}}\PY{l+s}{P}\PY{l+s}{\PYZdq{}}\PY{p}{,} \PY{l+s}{\PYZdq{}}\PY{l+s}{N}\PY{l+s}{\PYZdq{}}\PY{p}{]}
         
         \PY{k}{def} \PY{n+nf}{get\PYZus{}energy}\PY{p}{(}\PY{n}{point}\PY{p}{,} \PY{n}{coords}\PY{p}{)}\PY{p}{:}
             \PY{k}{if} \PY{n}{point} \PY{o}{==} \PY{l+s}{\PYZdq{}}\PY{l+s}{G}\PY{l+s}{\PYZdq{}}\PY{p}{:}
                 \PY{k}{return} \PY{p}{(}\PY{n}{coords}\PY{p}{[}\PY{l+m+mi}{0}\PY{p}{]}\PY{o}{\PYZhy{}}\PY{n}{coords}\PY{p}{[}\PY{l+m+mi}{1}\PY{p}{]}\PY{p}{)}\PY{o}{*}\PY{o}{*}\PY{l+m+mi}{2}\PY{o}{+}\PY{p}{(}\PY{n}{coords}\PY{p}{[}\PY{l+m+mi}{1}\PY{p}{]}\PY{o}{\PYZhy{}}\PY{n}{coords}\PY{p}{[}\PY{l+m+mi}{2}\PY{p}{]}\PY{p}{)}\PY{o}{*}\PY{o}{*}\PY{l+m+mi}{2}\PY{o}{+}\PY{p}{(}\PY{n}{coords}\PY{p}{[}\PY{l+m+mi}{0}\PY{p}{]}\PY{o}{+}\PY{n}{coords}\PY{p}{[}\PY{l+m+mi}{2}\PY{p}{]}\PY{p}{)}\PY{o}{*}\PY{o}{*}\PY{l+m+mi}{2}
             \PY{k}{elif} \PY{n}{point} \PY{o}{==} \PY{l+s}{\PYZdq{}}\PY{l+s}{H}\PY{l+s}{\PYZdq{}}\PY{p}{:}
                 \PY{k}{return} \PY{p}{(}\PY{n}{coords}\PY{p}{[}\PY{l+m+mi}{0}\PY{p}{]}\PY{o}{\PYZhy{}}\PY{n}{coords}\PY{p}{[}\PY{l+m+mi}{1}\PY{p}{]}\PY{o}{+}\PY{l+m+mf}{1.0}\PY{p}{)}\PY{o}{*}\PY{o}{*}\PY{l+m+mi}{2}\PY{o}{+}\PY{p}{(}\PY{n}{coords}\PY{p}{[}\PY{l+m+mi}{1}\PY{p}{]}\PY{o}{\PYZhy{}}\PY{n}{coords}\PY{p}{[}\PY{l+m+mi}{2}\PY{p}{]}\PY{p}{)}\PY{o}{*}\PY{o}{*}\PY{l+m+mi}{2}\PY{o}{+}\PY{p}{(}\PY{n}{coords}\PY{p}{[}\PY{l+m+mi}{0}\PY{p}{]}\PY{o}{+}\PY{n}{coords}\PY{p}{[}\PY{l+m+mi}{2}\PY{p}{]}\PY{p}{)}\PY{o}{*}\PY{o}{*}\PY{l+m+mi}{2}
             \PY{k}{elif} \PY{n}{point} \PY{o}{==} \PY{l+s}{\PYZdq{}}\PY{l+s}{P}\PY{l+s}{\PYZdq{}}\PY{p}{:}
                 \PY{k}{return} \PY{p}{(}\PY{n}{coords}\PY{p}{[}\PY{l+m+mi}{0}\PY{p}{]}\PY{o}{\PYZhy{}}\PY{n}{coords}\PY{p}{[}\PY{l+m+mi}{1}\PY{p}{]}\PY{o}{+}\PY{l+m+mf}{0.5}\PY{p}{)}\PY{o}{*}\PY{o}{*}\PY{l+m+mi}{2}\PY{o}{+}\PY{p}{(}\PY{n}{coords}\PY{p}{[}\PY{l+m+mi}{1}\PY{p}{]}\PY{o}{\PYZhy{}}\PY{n}{coords}\PY{p}{[}\PY{l+m+mi}{2}\PY{p}{]}\PY{o}{+}\PY{l+m+mf}{0.5}\PY{p}{)}\PY{o}{*}\PY{o}{*}\PY{l+m+mi}{2}\PY{o}{+}\PY{p}{(}\PY{n}{coords}\PY{p}{[}\PY{l+m+mi}{0}\PY{p}{]}\PY{o}{+}\PY{n}{coords}\PY{p}{[}\PY{l+m+mi}{2}\PY{p}{]}\PY{o}{+}\PY{l+m+mf}{0.5}\PY{p}{)}\PY{o}{*}\PY{o}{*}\PY{l+m+mi}{2}
             \PY{k}{elif} \PY{n}{point} \PY{o}{==} \PY{l+s}{\PYZdq{}}\PY{l+s}{N}\PY{l+s}{\PYZdq{}}\PY{p}{:}
                 \PY{k}{return} \PY{p}{(}\PY{n}{coords}\PY{p}{[}\PY{l+m+mi}{0}\PY{p}{]}\PY{o}{\PYZhy{}}\PY{n}{coords}\PY{p}{[}\PY{l+m+mi}{1}\PY{p}{]}\PY{o}{+}\PY{l+m+mf}{0.5}\PY{p}{)}\PY{o}{*}\PY{o}{*}\PY{l+m+mi}{2}\PY{o}{+}\PY{p}{(}\PY{n}{coords}\PY{p}{[}\PY{l+m+mi}{1}\PY{p}{]}\PY{o}{\PYZhy{}}\PY{n}{coords}\PY{p}{[}\PY{l+m+mi}{2}\PY{p}{]}\PY{o}{+}\PY{l+m+mf}{0.5}\PY{p}{)}\PY{o}{*}\PY{o}{*}\PY{l+m+mi}{2}\PY{o}{+}\PY{p}{(}\PY{n}{coords}\PY{p}{[}\PY{l+m+mi}{0}\PY{p}{]}\PY{o}{+}\PY{n}{coords}\PY{p}{[}\PY{l+m+mi}{2}\PY{p}{]}\PY{p}{)}\PY{o}{*}\PY{o}{*}\PY{l+m+mi}{2}
             \PY{k}{else}\PY{p}{:}
                 \PY{k}{print} \PY{l+s}{\PYZdq{}}\PY{l+s}{ERROR \PYZhy{} invalid point requested: }\PY{l+s}{\PYZdq{}}\PY{p}{,} \PY{n}{point}
         
         \PY{c}{\PYZsh{}recursive backtracking in order to get the valid reciprocal lattice points}
         \PY{k}{def} \PY{n+nf}{get\PYZus{}next}\PY{p}{(}\PY{n}{cand}\PY{p}{)}\PY{p}{:}
             \PY{n}{out} \PY{o}{=} \PY{p}{[}\PY{n}{elem} \PY{k}{for} \PY{n}{elem} \PY{o+ow}{in} \PY{n}{cand}\PY{p}{]}
             \PY{k}{if} \PY{n}{out}\PY{p}{[}\PY{l+m+mi}{2}\PY{p}{]}\PY{o}{==}\PY{l+m+mi}{3}\PY{p}{:}
                 \PY{k}{if} \PY{n}{out}\PY{p}{[}\PY{l+m+mi}{1}\PY{p}{]}\PY{o}{==}\PY{l+m+mi}{3}\PY{p}{:}
                     \PY{k}{if} \PY{n}{out}\PY{p}{[}\PY{l+m+mi}{0}\PY{p}{]}\PY{o}{==}\PY{l+m+mi}{3}\PY{p}{:}
                         \PY{k}{return} \PY{n+nb+bp}{None}
                     \PY{k}{else}\PY{p}{:}
                         \PY{n}{out}\PY{p}{[}\PY{l+m+mi}{0}\PY{p}{]}\PY{o}{+}\PY{o}{=}\PY{l+m+mi}{1}
                         \PY{n}{out}\PY{p}{[}\PY{l+m+mi}{1}\PY{p}{]}\PY{o}{=}\PY{o}{\PYZhy{}}\PY{l+m+mi}{3}
                         \PY{n}{out}\PY{p}{[}\PY{l+m+mi}{2}\PY{p}{]}\PY{o}{=}\PY{o}{\PYZhy{}}\PY{l+m+mi}{3}
                 \PY{k}{else}\PY{p}{:}
                     \PY{n}{out}\PY{p}{[}\PY{l+m+mi}{1}\PY{p}{]}\PY{o}{+}\PY{o}{=}\PY{l+m+mi}{1}
                     \PY{n}{out}\PY{p}{[}\PY{l+m+mi}{2}\PY{p}{]}\PY{o}{=}\PY{o}{\PYZhy{}}\PY{l+m+mi}{3}
             \PY{k}{else}\PY{p}{:}
                 \PY{n}{out}\PY{p}{[}\PY{l+m+mi}{2}\PY{p}{]}\PY{o}{+}\PY{o}{=}\PY{l+m+mi}{1}
             \PY{k}{return} \PY{n}{out}
         
         \PY{n}{h\PYZus{}list}\PY{o}{=}\PY{p}{[}\PY{p}{[}\PY{o}{\PYZhy{}}\PY{l+m+mi}{3}\PY{p}{,}\PY{o}{\PYZhy{}}\PY{l+m+mi}{3}\PY{p}{,}\PY{o}{\PYZhy{}}\PY{l+m+mi}{3}\PY{p}{]}\PY{p}{]}
         \PY{n}{cand} \PY{o}{=} \PY{n}{get\PYZus{}next}\PY{p}{(}\PY{n}{h\PYZus{}list}\PY{p}{[}\PY{o}{\PYZhy{}}\PY{l+m+mi}{1}\PY{p}{]}\PY{p}{)}
         \PY{k}{while} \PY{n}{cand} \PY{o}{!=} \PY{n+nb+bp}{None}\PY{p}{:}
             \PY{n}{h\PYZus{}list}\PY{o}{.}\PY{n}{append}\PY{p}{(}\PY{p}{[}\PY{n}{elem} \PY{k}{for} \PY{n}{elem} \PY{o+ow}{in} \PY{n}{cand}\PY{p}{]}\PY{p}{)}
             \PY{n}{cand} \PY{o}{=} \PY{n}{get\PYZus{}next}\PY{p}{(}\PY{n}{h\PYZus{}list}\PY{p}{[}\PY{o}{\PYZhy{}}\PY{l+m+mi}{1}\PY{p}{]}\PY{p}{)}
         
         \PY{n}{all\PYZus{}energies} \PY{o}{=} \PY{p}{\PYZob{}}\PY{p}{\PYZcb{}}
         \PY{n}{all\PYZus{}occurences} \PY{o}{=} \PY{p}{\PYZob{}}\PY{p}{\PYZcb{}}
         
         \PY{c}{\PYZsh{}build the dictionnaries containing possible energy levels}
         \PY{c}{\PYZsh{}and their degeneracy}
         \PY{k}{for} \PY{n}{point} \PY{o+ow}{in} \PY{n}{special\PYZus{}points}\PY{p}{:}
             \PY{n}{all\PYZus{}energies}\PY{p}{[}\PY{n}{point}\PY{p}{]}\PY{o}{=}\PY{p}{[}\PY{p}{]}
             \PY{k}{for} \PY{n}{elem} \PY{o+ow}{in} \PY{n}{h\PYZus{}list}\PY{p}{:}
                 \PY{n}{tt} \PY{o}{=} \PY{n}{get\PYZus{}energy}\PY{p}{(}\PY{n}{point}\PY{p}{,} \PY{n}{elem}\PY{p}{)}
                 \PY{n}{all\PYZus{}energies}\PY{p}{[}\PY{n}{point}\PY{p}{]}\PY{o}{.}\PY{n}{append}\PY{p}{(}\PY{n}{tt}\PY{p}{)}
                 \PY{k}{if} \PY{p}{(}\PY{n}{point}\PY{p}{,} \PY{n}{tt}\PY{p}{)} \PY{o+ow}{in} \PY{n}{all\PYZus{}occurences}\PY{p}{:}
                     \PY{n}{all\PYZus{}occurences}\PY{p}{[}\PY{p}{(}\PY{n}{point}\PY{p}{,} \PY{n}{tt}\PY{p}{)}\PY{p}{]}\PY{o}{.}\PY{n}{append}\PY{p}{(}\PY{n}{elem}\PY{p}{)}
                 \PY{k}{else}\PY{p}{:}
                     \PY{n}{all\PYZus{}occurences}\PY{p}{[}\PY{p}{(}\PY{n}{point}\PY{p}{,} \PY{n}{tt}\PY{p}{)}\PY{p}{]} \PY{o}{=} \PY{p}{[}\PY{n}{elem}\PY{p}{]}
         
         
         \PY{c}{\PYZsh{}display(Math(r\PYZsq{}F(k) = \PYZbs{}int\PYZus{}\PYZob{}\PYZhy{}\PYZbs{}infty\PYZcb{}\PYZca{}\PYZob{}\PYZbs{}infty\PYZcb{} f(x) e\PYZca{}\PYZob{}2\PYZbs{}pi i k\PYZcb{} dx\PYZsq{}))}
         \PY{k}{print} \PY{l+s}{\PYZdq{}}\PY{l+s+se}{\PYZbs{}n}\PY{l+s}{\PYZdq{}}
         \PY{k}{print} \PY{l+s}{\PYZdq{}}\PY{l+s+se}{\PYZbs{}n}\PY{l+s}{\PYZdq{}}
         \PY{n}{display}\PY{p}{(}\PY{n}{Latex}\PY{p}{(}\PY{l+s}{r\PYZsq{}}\PY{l+s}{Energy levels per special point, in units of \PYZdl{}(}\PY{l+s}{\PYZbs{}}\PY{l+s}{hbar\PYZca{}2/2m)(2}\PY{l+s}{\PYZbs{}}\PY{l+s}{pi/a)\PYZca{}2\PYZdl{}}\PY{l+s}{\PYZsq{}}\PY{p}{)}\PY{p}{)}
         \PY{k}{for} \PY{n}{point} \PY{o+ow}{in} \PY{n}{special\PYZus{}points}\PY{p}{:}
             \PY{k}{print} \PY{l+s}{\PYZdq{}}\PY{l+s}{point: }\PY{l+s}{\PYZdq{}}\PY{p}{,} \PY{n}{point}
             \PY{n}{valid\PYZus{}energies} \PY{o}{=} \PY{n+nb}{list}\PY{p}{(}\PY{n+nb}{set}\PY{p}{(}\PY{n}{all\PYZus{}energies}\PY{p}{[}\PY{n}{point}\PY{p}{]}\PY{p}{)}\PY{p}{)}
             \PY{n}{valid\PYZus{}energies}\PY{o}{.}\PY{n}{sort}\PY{p}{(}\PY{p}{)}
             \PY{k}{for} \PY{n}{e} \PY{o+ow}{in} \PY{n}{valid\PYZus{}energies}\PY{p}{:}
                 \PY{k}{if} \PY{n}{e}\PY{o}{\PYZlt{}}\PY{o}{=}\PY{l+m+mi}{9}\PY{p}{:}
                     \PY{k}{print} \PY{l+s}{\PYZdq{}}\PY{l+s}{E = }\PY{l+s}{\PYZdq{}}\PY{p}{,} \PY{n}{e}\PY{p}{,} \PY{l+s}{\PYZdq{}}\PY{l+s}{; degeneracy: }\PY{l+s}{\PYZdq{}}\PY{p}{,} \PY{n}{all\PYZus{}energies}\PY{p}{[}\PY{n}{point}\PY{p}{]}\PY{o}{.}\PY{n}{count}\PY{p}{(}\PY{n}{e}\PY{p}{)}
         
         \PY{k}{print} \PY{l+s}{\PYZdq{}}\PY{l+s+se}{\PYZbs{}n}\PY{l+s}{\PYZdq{}}
                  
         \PY{k}{print} \PY{l+s}{\PYZdq{}}\PY{l+s}{A few examples of degeneracy (h1, h2, h3 triplets associated with a special point/energy couple)}\PY{l+s}{\PYZdq{}}
         \PY{k}{print} \PY{l+s}{\PYZdq{}}\PY{l+s}{point N, E=1.5: }\PY{l+s}{\PYZdq{}}\PY{p}{,} \PY{n}{all\PYZus{}occurences}\PY{p}{[}\PY{p}{(}\PY{l+s}{\PYZdq{}}\PY{l+s}{N}\PY{l+s}{\PYZdq{}}\PY{p}{,} \PY{l+m+mf}{1.5}\PY{p}{)}\PY{p}{]}
         \PY{k}{print} \PY{l+s}{\PYZdq{}}\PY{l+s}{point G, E=2: }\PY{l+s}{\PYZdq{}}\PY{p}{,} \PY{n}{all\PYZus{}occurences}\PY{p}{[}\PY{p}{(}\PY{l+s}{\PYZdq{}}\PY{l+s}{G}\PY{l+s}{\PYZdq{}}\PY{p}{,} \PY{l+m+mi}{2}\PY{p}{)}\PY{p}{]}
         \PY{k}{print} \PY{l+s}{\PYZdq{}}\PY{l+s+se}{\PYZbs{}n}\PY{l+s}{\PYZdq{}}
\end{Verbatim}

    Energy levels per special point, in units of $(\hbar^2/2m)(2\pi/a)^2$

    
    \begin{Verbatim}[commandchars=\\\{\}]
point:  G
E =  0 ; degeneracy:  1
E =  2 ; degeneracy:  12
E =  4 ; degeneracy:  6
E =  6 ; degeneracy:  24
E =  8 ; degeneracy:  12
point:  H
E =  1.0 ; degeneracy:  6
E =  3.0 ; degeneracy:  8
E =  5.0 ; degeneracy:  24
E =  9.0 ; degeneracy:  30
point:  P
E =  0.75 ; degeneracy:  4
E =  2.75 ; degeneracy:  12
E =  4.75 ; degeneracy:  12
E =  6.75 ; degeneracy:  16
E =  8.75 ; degeneracy:  24
point:  N
E =  0.5 ; degeneracy:  2
E =  1.5 ; degeneracy:  4
E =  2.5 ; degeneracy:  4
E =  3.5 ; degeneracy:  8
E =  4.5 ; degeneracy:  6
E =  5.5 ; degeneracy:  4
E =  6.5 ; degeneracy:  12
E =  7.5 ; degeneracy:  8
E =  8.5 ; degeneracy:  8


A few examples of degeneracy (h1, h2, h3 triplets associated with a special point/energy couple)
point N, E=1.5:  [[-1, -1, 0], [-1, 0, 0], [0, 0, 1], [0, 1, 1]]
point G, E=2:  [[-1, -1, 0], [-1, 0, 0], [-1, 0, 1], [0, -1, -1], [0, -1, 0], [0, 0, -1], [0, 0, 1], [0, 1, 0], [0, 1, 1], [1, 0, -1], [1, 0, 0], [1, 1, 0]]
    \end{Verbatim}

    \begin{Verbatim}[commandchars=\\\{\}]
{\color{incolor}In [{\color{incolor}56}]:} \PY{k+kn}{import} \PY{n+nn}{numpy} \PY{k+kn}{as} \PY{n+nn}{np}
         
         \PY{n}{fig} \PY{o}{=} \PY{n}{plt}\PY{o}{.}\PY{n}{figure}\PY{p}{(}\PY{n}{figsize}\PY{o}{=}\PY{p}{(}\PY{l+m+mi}{6}\PY{o}{*}\PY{l+m+mi}{3}\PY{p}{,}\PY{l+m+mi}{10}\PY{p}{)}\PY{p}{)}
         \PY{n}{ymin} \PY{o}{=} \PY{l+m+mi}{0}
         \PY{n}{ymax} \PY{o}{=} \PY{l+m+mi}{3}
         \PY{n}{xmin} \PY{o}{=} \PY{l+m+mi}{0}
         \PY{n}{xmax} \PY{o}{=} \PY{l+m+mi}{1}
         \PY{n}{x\PYZus{}values} \PY{o}{=} \PY{n}{np}\PY{o}{.}\PY{n}{linspace}\PY{p}{(}\PY{l+m+mi}{0}\PY{p}{,} \PY{l+m+mi}{1}\PY{p}{,} \PY{n}{num}\PY{o}{=}\PY{l+m+mi}{100}\PY{p}{)}
         \PY{n}{titles} \PY{o}{=} \PY{p}{[}\PY{l+s}{r\PYZsq{}}\PY{l+s}{H \PYZhy{} \PYZdl{}}\PY{l+s}{\PYZbs{}}\PY{l+s}{Gamma\PYZdl{}}\PY{l+s}{\PYZsq{}}\PY{p}{,} \PY{l+s}{r\PYZsq{}}\PY{l+s}{\PYZdl{}}\PY{l+s}{\PYZbs{}}\PY{l+s}{Gamma\PYZdl{} \PYZhy{} N}\PY{l+s}{\PYZsq{}}\PY{p}{,} \PY{l+s}{r\PYZsq{}}\PY{l+s}{N\PYZhy{}P}\PY{l+s}{\PYZsq{}}\PY{p}{]}
         
         \PY{n}{starts}\PY{o}{=}\PY{p}{[}\PY{p}{[}\PY{p}{[}\PY{p}{(}\PY{l+m+mi}{0}\PY{p}{,}\PY{l+m+mi}{1}\PY{p}{)}\PY{p}{]}\PY{p}{,} \PY{p}{[}\PY{p}{(}\PY{l+m+mi}{0}\PY{p}{,}\PY{l+m+mi}{1}\PY{p}{)}\PY{p}{,} \PY{p}{(}\PY{l+m+mi}{0}\PY{p}{,}\PY{l+m+mi}{3}\PY{p}{)}\PY{p}{,} \PY{p}{(}\PY{l+m+mi}{0}\PY{p}{,}\PY{l+m+mi}{5}\PY{p}{)}\PY{p}{]}\PY{p}{,} \PY{p}{[}\PY{p}{(}\PY{l+m+mi}{0}\PY{p}{,}\PY{l+m+mi}{1}\PY{p}{)}\PY{p}{]}\PY{p}{]}\PY{p}{,} \PY{p}{[}\PY{p}{[}\PY{p}{(}\PY{l+m+mi}{0}\PY{p}{,}\PY{l+m+mi}{0}\PY{p}{)}\PY{p}{]}\PY{p}{,} \PY{p}{[}\PY{p}{(}\PY{l+m+mi}{0}\PY{p}{,} \PY{l+m+mi}{2}\PY{p}{)}\PY{p}{]}\PY{p}{,} \PY{p}{[}\PY{p}{(}\PY{l+m+mi}{0}\PY{p}{,}\PY{l+m+mi}{4}\PY{p}{)}\PY{p}{]}\PY{p}{]}\PY{p}{,} \PY{p}{[}\PY{p}{[}\PY{p}{(}\PY{l+m+mi}{0}\PY{p}{,} \PY{l+m+mf}{0.5}\PY{p}{)}\PY{p}{]}\PY{p}{,} \PY{p}{[}\PY{p}{(}\PY{l+m+mi}{0}\PY{p}{,} \PY{l+m+mf}{1.5}\PY{p}{)}\PY{p}{]}\PY{p}{,} \PY{p}{[}\PY{p}{(}\PY{l+m+mi}{0}\PY{p}{,} \PY{l+m+mf}{2.5}\PY{p}{)}\PY{p}{]}\PY{p}{,} \PY{p}{[}\PY{p}{(}\PY{l+m+mi}{0}\PY{p}{,} \PY{l+m+mf}{3.5}\PY{p}{)}\PY{p}{]}\PY{p}{]}\PY{p}{]}
         \PY{n}{ends}\PY{o}{=}\PY{p}{[}\PY{p}{[}\PY{p}{[}\PY{p}{(}\PY{l+m+mi}{1}\PY{p}{,} \PY{l+m+mi}{0}\PY{p}{)}\PY{p}{]}\PY{p}{,} \PY{p}{[}\PY{p}{(}\PY{l+m+mi}{1}\PY{p}{,}\PY{l+m+mi}{2}\PY{p}{)}\PY{p}{]}\PY{p}{,} \PY{p}{[}\PY{p}{(}\PY{l+m+mi}{1}\PY{p}{,}\PY{l+m+mi}{4}\PY{p}{)}\PY{p}{]}\PY{p}{]}\PY{p}{,}\PY{p}{[}\PY{p}{[}\PY{p}{(}\PY{l+m+mi}{1}\PY{p}{,}\PY{l+m+mf}{0.5}\PY{p}{)}\PY{p}{]}\PY{p}{,} \PY{p}{[}\PY{p}{(}\PY{l+m+mi}{1}\PY{p}{,} \PY{l+m+mf}{0.5}\PY{p}{)}\PY{p}{,} \PY{p}{(}\PY{l+m+mi}{1}\PY{p}{,}\PY{l+m+mf}{1.5}\PY{p}{)}\PY{p}{,} \PY{p}{(}\PY{l+m+mi}{1}\PY{p}{,} \PY{l+m+mf}{2.5}\PY{p}{)}\PY{p}{,} \PY{p}{(}\PY{l+m+mi}{1}\PY{p}{,} \PY{l+m+mf}{3.5}\PY{p}{)}\PY{p}{,} \PY{p}{(}\PY{l+m+mi}{1}\PY{p}{,} \PY{l+m+mf}{4.5}\PY{p}{)}\PY{p}{]}\PY{p}{,} \PY{p}{[}\PY{p}{(}\PY{l+m+mi}{1}\PY{p}{,} \PY{l+m+mf}{2.5}\PY{p}{)}\PY{p}{]}\PY{p}{]}\PY{p}{,}\PY{p}{[}\PY{p}{[}\PY{p}{(}\PY{l+m+mi}{1}\PY{p}{,} \PY{l+m+mf}{0.75}\PY{p}{)}\PY{p}{]}\PY{p}{,} \PY{p}{[}\PY{p}{(}\PY{l+m+mi}{1}\PY{p}{,} \PY{l+m+mf}{0.75}\PY{p}{)}\PY{p}{,} \PY{p}{(}\PY{l+m+mi}{1}\PY{p}{,} \PY{l+m+mf}{2.75}\PY{p}{)}\PY{p}{]}\PY{p}{,} \PY{p}{[}\PY{p}{(}\PY{l+m+mi}{1}\PY{p}{,} \PY{l+m+mf}{2.75}\PY{p}{)}\PY{p}{]}\PY{p}{,} \PY{p}{[}\PY{p}{(}\PY{l+m+mi}{1}\PY{p}{,} \PY{l+m+mf}{2.75}\PY{p}{)}\PY{p}{]}\PY{p}{]}\PY{p}{]}
         
         \PY{k}{for} \PY{n}{plot\PYZus{}idx} \PY{o+ow}{in} \PY{n+nb}{range}\PY{p}{(}\PY{l+m+mi}{3}\PY{p}{)}\PY{p}{:}
             \PY{n}{axes} \PY{o}{=} \PY{n}{fig}\PY{o}{.}\PY{n}{add\PYZus{}subplot}\PY{p}{(}\PY{l+m+mi}{1}\PY{p}{,}\PY{l+m+mi}{3}\PY{p}{,}\PY{l+m+mi}{1}\PY{o}{+}\PY{n}{plot\PYZus{}idx}\PY{p}{)}
             \PY{k}{for} \PY{n}{idx}\PY{p}{,} \PY{n}{a} \PY{o+ow}{in} \PY{n+nb}{enumerate}\PY{p}{(}\PY{n}{starts}\PY{p}{[}\PY{n}{plot\PYZus{}idx}\PY{p}{]}\PY{p}{)}\PY{p}{:}
                 \PY{k}{for} \PY{n}{b\PYZus{}elem} \PY{o+ow}{in} \PY{n}{ends}\PY{p}{[}\PY{n}{plot\PYZus{}idx}\PY{p}{]}\PY{p}{[}\PY{n}{idx}\PY{p}{]}\PY{p}{:}
                     \PY{k}{for} \PY{n}{a\PYZus{}elem} \PY{o+ow}{in} \PY{n}{a}\PY{p}{:}
                         \PY{n}{axes}\PY{o}{.}\PY{n}{plot}\PY{p}{(}\PY{n}{x\PYZus{}values}\PY{p}{,} \PY{p}{[}\PY{n}{a\PYZus{}elem}\PY{p}{[}\PY{l+m+mi}{1}\PY{p}{]}\PY{o}{+}\PY{p}{(}\PY{n}{b\PYZus{}elem}\PY{p}{[}\PY{l+m+mi}{1}\PY{p}{]}\PY{o}{\PYZhy{}}\PY{n}{a\PYZus{}elem}\PY{p}{[}\PY{l+m+mi}{1}\PY{p}{]}\PY{p}{)}\PY{o}{*}\PY{n}{zz} \PY{k}{for} \PY{n}{zz} \PY{o+ow}{in} \PY{n}{x\PYZus{}values}\PY{p}{]}\PY{p}{,} \PY{n}{color}\PY{o}{=}\PY{l+s}{\PYZsq{}}\PY{l+s}{k}\PY{l+s}{\PYZsq{}}\PY{p}{,} \PY{n}{linestyle}\PY{o}{=}\PY{l+s}{\PYZsq{}}\PY{l+s}{\PYZhy{}}\PY{l+s}{\PYZsq{}}\PY{p}{,} \PY{n}{linewidth}\PY{o}{=}\PY{l+m+mi}{2}\PY{p}{)}
             \PY{c}{\PYZsh{}axes.set\PYZus{}xlabel(\PYZsq{}\PYZdl{}k\PYZdl{}(Angstrom)\PYZsq{}, fontsize=18)}
             \PY{c}{\PYZsh{}axes.set\PYZus{}ylabel(\PYZsq{}\PYZdl{}\PYZbs{}omega\PYZdl{} (meV)\PYZsq{}, fontsize=18)}
             \PY{n}{axes}\PY{o}{.}\PY{n}{set\PYZus{}title}\PY{p}{(}\PY{n}{titles}\PY{p}{[}\PY{n}{plot\PYZus{}idx}\PY{p}{]}\PY{p}{,} \PY{n}{fontsize}\PY{o}{=}\PY{l+m+mi}{20}\PY{p}{)}
             \PY{c}{\PYZsh{}axes.legend(loc=2) \PYZsh{} upper left corner}
             \PY{n}{axes}\PY{o}{.}\PY{n}{set\PYZus{}yticks}\PY{p}{(}\PY{n}{np}\PY{o}{.}\PY{n}{linspace}\PY{p}{(}\PY{n}{ymin}\PY{p}{,}\PY{n}{ymax}\PY{p}{,}\PY{l+m+mi}{1}\PY{o}{+}\PY{n}{ymax}\PY{o}{\PYZhy{}}\PY{n}{ymin}\PY{p}{,}\PY{n}{endpoint}\PY{o}{=}\PY{n+nb+bp}{True}\PY{p}{)}\PY{p}{)}
             \PY{n}{axes}\PY{o}{.}\PY{n}{set\PYZus{}xticks}\PY{p}{(}\PY{p}{[}\PY{l+m+mi}{0}\PY{p}{,}\PY{l+m+mi}{1}\PY{p}{]}\PY{p}{)}
             \PY{n}{axes}\PY{o}{.}\PY{n}{set\PYZus{}ylim}\PY{p}{(}\PY{p}{[}\PY{n}{ymin}\PY{p}{,}\PY{n}{ymax}\PY{p}{]}\PY{p}{)}
             \PY{n}{axes}\PY{o}{.}\PY{n}{set\PYZus{}xlim}\PY{p}{(}\PY{p}{[}\PY{n}{xmin}\PY{p}{,}\PY{n}{xmax}\PY{p}{]}\PY{p}{)}
             \PY{n}{axes}\PY{o}{.}\PY{n}{grid}\PY{p}{(}\PY{n+nb+bp}{True}\PY{p}{)}
\end{Verbatim}

    \begin{center}
    \adjustimage{max size={0.9\linewidth}{0.9\paperheight}}{Free_electrons_in_Na_files/Free_electrons_in_Na_21_0.png}
    \end{center}
    { \hspace*{\fill} \\}
    

    % Add a bibliography block to the postdoc
    
    
    
    \end{document}
